\documentclass[12pt]{article}

% packages
\usepackage{enumitem} % enumerations
\usepackage{fancyhdr} % header
\usepackage[a4paper, margin=2.5cm]{geometry} % customize page
\usepackage[colorlinks=true]{hyperref} % links
\usepackage{xcolor} % colors

% header
\pagestyle{fancy}
\fancyhead[L]{\textbf{Exercise Sheet 3: Association Rules}}
\fancyhead[R]{Data Science 1 -- Winter 21/22}

% colors and commands
\definecolor{kitgreen}{HTML}{00876C}
\definecolor{kitblue}{HTML}{4664AA}

\hypersetup{allcolors=kitblue}

\newcommand{\code}[1]{\textcolor{kitgreen}{\texttt{#1}}}
\newcommand{\taskname}[1]{\textcolor{kitblue}{\textbf{[#1]}}}

\begin{document}

\section*{Setup}

Compared to the previous exercise sheets, you should install \code{mlxtend}, \code{pyreadr}, and \code{rdata}.
Besides that, we will work with \code{matplotlib} and \code{pandas}.

\section*{Task: Association Rules}

The aim of this exercise is to apply different variants of frequent itemset mining and association rule mining.
We work with the \code{Groceries} dataset, which you can obtain by running the script \code{prepare\_groceries\_dataset.py} provided on ILIAS.
The \href{https://rdrr.io/cran/arules/man/Groceries.html}{dataset} contains real-world transaction data from a grocery outlet.

\begin{enumerate}[label=\alph*), left=0pt, itemsep=12pt]
	\item
	\taskname{Transaction Data}
	Load the dataset and bring it into a form suitable for analysis.
	Python's built-in \code{open()} routine in combination with some simple string operations should suffice.
	\newline
	How is the dataset structured?
	How many different items are there?
	How is the length of transactions distributed?
	How is the frequency of items distributed?
	\item
	\taskname{Frequent Itemset Mining}
	Use \code{apriori()} from \code{mlxtend.frequent\_patterns} to determine all frequent itemsets with a support of at least 5\%.
	\code{apriori()} requires the transaction data to be in a specifically encoded \code{pandas.DataFrame}.
	You can use a \code{TransactionEncoder} from \code{mlxtend.preprocessing} for that purpose.
	\newline
	Are all of the frequent itemsets also maximally frequent?
	Why / why not?
	\item
	\taskname{Association Rules Mining}
	Use functions \code{apriori()} and \code{association\_rules()} from \code{mlxtend.frequent\_patterns} to determine all association rules with a support of at least 1\% and a minimum confidence of 40\%.
	\newline
	Which five rules have the highest confidence?
	Which rules contain \textit{yogurt} (as one of the items) on the left-hand side and have a confidence greater than 50\%?
	\item
	\taskname{Multi-Level Mining}
	Use the mapping contained in \code{groceries\_structure.csv} to aggregate the grocery data to \code{level2}.
	In other words, replace all \code{label} values in the initial groceries dataset with their corresponding \code{level2} value.
	Make sure to avoid duplicate items in each transaction.
	Extract all rules with a minimum support of 10\% and a minimum confidence of 40\%.
	\newline
	Why is it reasonable to use a higher support threshold than in the previous subtasks?
	\item
	\taskname{Level-Crossing Mining}
	Create a level-crossing representation of the groceries dataset, including the original \code{label} values besides their respective \code{level2} values.
	This means there should be two `items' for each actual item now.
	Make sure this also is the case if there are naming clashes between the two levels.
	Extract association rules with the previous subtask's thresholds.
	\newline
	Which challenges do you encounter due to the level-crossing representation?
\end{enumerate}

\end{document}
